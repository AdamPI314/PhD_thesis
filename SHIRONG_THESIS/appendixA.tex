\chapter{The Hydrogen Combustion Model}	% *NOT* \OnePageChapter
\label{appendixC}
The combustion of H$_2$ is modeled using a detailed mechanism that consists of 21 individual steps and 8 distinct chemical species.  The specific mechanism employed here is a sub-mechanism of a larger model for methanol combustion used by Li et al.\cite{aHm_1_li2007comprehensive} The reactions and species used are listed in the tables below. The temperature dependence of the rate coefficients is given by the three parameter fit
\begin{equation}
\label{aHm:eqn1}
k(T) = AT^{\delta}E^{-E_a/{k_BT}}
\end{equation}
where the values of $A$, $\delta$, and $E_a$ for each reaction are listed in Table \ref{aHm:table1}.  The pressure-dependent reactions rate are calculated using the Troe centering expression
\begin{equation}
\label{aHm:eqn2}
k=k_{\infty \left( \frac{P_r}{1+P_r}F \right) }
\end{equation}
where
\begin{equation}
\label{aHm:eqn3}
P_r = \frac{k_0 \left[ M \right] }{k_{\infty}}
\end{equation}
and the high pressure limit ($k_{\infty}$) and low pressure limit ($k_0$) rate coefficients are parameterized by the three parameter expression in eqn. \ref{aHm:eqn1}.  The function F \footnote{\label{Fdef}  The function F is defined as $logF= {\left[  1+ \left[  \frac{logP_r + c}{n - d(logP_r+c)} \right] \right]}^{-1} logF_{cent} $ where $F_{cent} = (1-\alpha)exp(-T/T^{\ast\ast\ast})+\alpha exp(T/T^{\ast})+exp(-T^{\ast\ast}/T)$ and centering parameters $T^{\ast}$, $T^{\ast\ast}$, and $T^{\ast\ast\ast}$ are provided in the table.} is given by Troe and coworkers.\cite{aHm_2_gilbert1983theory} The reverse reaction rates are defined consistently with micro-reversibility using the equilibrium constant
\begin{equation}
\label{aHm:eqn4}
k_r = \frac{k_f}{k_{eq}^{c}}
\end{equation}
In the mechanism, there are two pairs of duplicated reactions; i.e., they have the same reactants and products but different rate coefficients. This allows a much more flexible representation of the overall temperature dependence of the rate coefficients.  
The combustion kinetics is simulated using the constraints of constant volume and constant energy.  The time development of each species is modeled by the following set of equations:
\begin{equation}
\label{aHm:eqn5}
\begin{split}
\frac{dY_k}{dt} &= \frac{W_k \dot{\omega}_k}{\rho},~~~k=1,\cdots,8 \\
\frac{dT}{dt} &= -\frac{1}{\rho C_v} \sum_k{W_k \dot{\omega}_ke_k}
\end{split}
\end{equation}
where $Y_k$ represents the mass fraction of species k (which is proportional to its concentration $\left[ X_k\right]$), $W_k$ is its molecular weight, $\dot{\omega}_k$ refers to the production rate of the kth species obtained from the mechanism, $\rho$ is the bulk density, $C_v$ is the constant volume heat capacity and $e_k$ is the internal energy of the kth species.

\begin{table}[htb]
    \caption[Rate coefficients and uncertainty ranges of H$_2$-O$_2$ system]{Rate coefficients and uncertainty ranges of H$_2$-O$_2$ system.}
    \begin{center}
    \begin{tabular}{|c|c|c|c|c|c|}
\hline
Index  & Reaction            & A(cm$^{3}$/mol-s) or s$^{-1}$         & $\delta$     & E$_a$(kcal/mol)         & Uncertainty \\ \hline
0      & H+O$_2$=O+OH           & 3.55E+15  & -0.4         & 16599.0       & 1.26        \\ \hline
1      & O+H$_2$=H+OH           & 5.08E+04  & 2.7          & 6290.0        & 1.58        \\ \hline
2      & H$_2$+OH=H$_2$O+H         & 2.16E+08  & 1.5          & 3430.0        & 2.00        \\ \hline
3      & O+H$_2$O=OH+OH         & 2.97E+06  & 2.0          & 13400.0       & 2.50        \\ \hline
4      & H$_2$+M=H+H+M          & 4.58E+19  & -1.4         & 104380.0      & 3.00        \\ \hline
\multicolumn{3}{|c|}{H$_2$}                 & \multicolumn{3}{l|}{Enhanced by 2.500E+00} \\ \hline
\multicolumn{3}{|c|}{H$_2$O}                & \multicolumn{3}{l|}{Enhanced by 1.200E+01} \\ \hline
5      & O+O+M=O$_2$+M          & 6.16E+15  & -0.5         & 0.0           & 2.00        \\ \hline
\multicolumn{3}{|c|}{H$_2$}                 & \multicolumn{3}{l|}{Enhanced by 2.500E+00} \\ \hline
\multicolumn{3}{|c|}{H$_2$O}                & \multicolumn{3}{l|}{Enhanced by 1.200E+01} \\ \hline
6      & O+H+M=OH+M          & 4.71E+18  & -1.0         & 0.0           & 5.00        \\ \hline
\multicolumn{3}{|c|}{H$_2$}                 & \multicolumn{3}{l|}{Enhanced by 2.500E+00} \\ \hline
\multicolumn{3}{|c|}{H$_2$O}                & \multicolumn{3}{l|}{Enhanced by 1.200E+01} \\ \hline
7      & H+OH+M=H$_2$O+M        & 3.80E+22  & -2.0         & 0.0           & 2.00        \\ \hline
\multicolumn{3}{|c|}{H$_2$}                 & \multicolumn{3}{l|}{Enhanced by 2.500E+00} \\ \hline
\multicolumn{3}{|c|}{H$_2$O}                & \multicolumn{3}{l|}{Enhanced by,1.200E+01} \\ \hline
8      & H+O$_2$(+M)=HO$_2$(+M)    & 1.48E+12  & 0.6          & 0.0           & 3.16        \\ \hline
\multicolumn{3}{|c|}{Low pressure limit} & 0.63660E+21  & -0.17200E+01  & 0.52480E+03 \\ \hline
\multicolumn{3}{|c|}{TROE centering}     & 0.80000E+00  & 0.10000E-29   & 0.10000E+31 \\ \hline
\multicolumn{3}{|c|}{H$_2$}                 & \multicolumn{3}{l|}{Enhanced by 2.000E+00} \\ \hline
\multicolumn{3}{|c|}{H$_2$O}                & \multicolumn{3}{l|}{Enhanced by 1.100E+01} \\ \hline
\multicolumn{3}{|c|}{O$_2$}                 & \multicolumn{3}{l|}{Enhanced by 7.800E-01} \\ \hline
9      & HO$_2$+H=H$_2$+O$_2$         & 1.66E+13  & 0.0          & 823.0         & 2.00        \\ \hline
10     & HO$_2$+H=OH+OH         & 7.08E+13  & 0.0          & 295.0         & 2.00        \\ \hline
11     & HO$_2$+O=O$_2$+OH         & 3.25E+13  & 0.0          & 0.0           & 3.16        \\ \hline
12     & HO$_2$+OH=H$_2$O+O$_2$       & 2.89E+13  & 0.0          & -497.0        & 3.16        \\ \hline
13     & HO$_2$+HO$_2$=H$_2$O$_2$+O$_2$     & 4.20E+14  & 0.0          & 11982.0       & 5.00        \\ \hline
\multicolumn{6}{|c|}{Declared duplicate reaction...}                                  \\ \hline
14     & HO$_2$+HO$_2$=H$_2$O$_2$+O$_2$     & 1.30E+11  & 0.0          & -1629.3       & 5.00        \\ \hline
\multicolumn{6}{|c|}{Declared duplicate reaction...}                                  \\ \hline
15     & H$_2$O$_2$(+M)=OH+OH(+M)  & 2.95E+14  & 0.0          & 48430.0       & 3.16        \\ \hline
\multicolumn{3}{|c|}{Low pressure limit} & 0.12020E+18  & 0.00000E+00   & 0.45500E+05 \\ \hline
\multicolumn{3}{|c|}{TROE centering}     & 0.50000E+00  & 0.10000E-29   & 0.10000E+31 \\ \hline
\multicolumn{3}{|c|}{H$_2$}                 & \multicolumn{3}{l|}{Enhanced by 2.500E+00} \\ \hline
\multicolumn{3}{|c|}{H$_2$O}                & \multicolumn{3}{l|}{Enhanced by 1.200E+01} \\ \hline
16     & H$_2$O$_2$+H=H$_2$O+OH       & 2.41E+13  & 0.0          & 3970.0        & 5.00        \\ \hline
17     & H$_2$O$_2$+H=HO$_2$+H$_2$       & 4.82E+13  & 0.0          & 7950.0        & 5.00        \\ \hline
18     & H$_2$O$_2$+O=OH+HO$_2$       & 9.55E+06  & 2.0          & 3970.0        & 3.00        \\ \hline
19     & H$_2$O$_2$+OH=HO$_2$+H$_2$O     & 1.00E+12  & 0.0          & 0.0           & 5.00        \\ \hline
\multicolumn{6}{|c|}{Declared duplicate reaction...}                                  \\ \hline
20     & H$_2$O$_2$+OH=HO$_2$+H$_2$O     & 5.80E+14  & 0.0          & 9557.0        & 5.00        \\ \hline
\multicolumn{6}{|c|}{Declared duplicate reaction...}                                  \\ \hline
\end{tabular}
   \\ \rule{0mm}{5mm}
   ${}^\dagger$Indices start at 0.		% footnote symbol
\end{center}
\label{aHm:table1}
\end{table}