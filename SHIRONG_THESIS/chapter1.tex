\chapter{Introduction}
\label{introchap}
Many problems of great practical importance are described by large interconnected networks
of elementary chemical reactions.\cite{ch1_IRPC_1_laidler1987chemical,ch1_IRPC_2_pilling1996reaction,ch1_IRPC_3_steinfeld1989chemical} In combustion chemistry, for example,
numerical simulations combine chemical mechanisms with computational fluid dynamics
to describe the consumption of fuels under engine operating conditions.\cite{ch1_IRPC_4_warnatz2017combustion,ch1_IRPC_5_westbrook2000chemical,ch1_IRPC_6_miller1990chemical,ch1_IRPC_7_battin2008detailed,ch1_IRPC_8_zador2011kinetics} These
mechanisms can grow quite large, e.g. n-heptane combustion can be modelled by a network
of 7000 elementary gas phase reactions.\cite{ch1_IRPC_9_curran1998comprehensive} Likewise, the CAM-chem model of
tropospheric chemistry can potentially consist of thousands of elementary chemical
reactions.\cite{ch1_IRPC_10_lamarque2012cam} Reactions on surfaces provide another class of problems modelled by
large chemical mechanisms.\cite{ch1_IRPC_11_somorjai2010introduction,ch1_IRPC_12_hammer2000theoretical} The accurate performance of these models is necessary
for important objectives such as engine design\cite{ch1_IRPC_13_som2013quantum} and climate prediction.\cite{ch1_IRPC_14_lamarque2013atmospheric}
The construction of accurate chemical mechanisms is typically an iterative endeavour
where the model is repeatedly improved after comparison to experiment. This process
requires insight into the workings of the chemical network. Unfortunately, the behaviour
of complex kinetic mechanisms can become quite obscure making the process of
validation and improvement difficult. Even the task of numerical simulation can
become computationally prohibitive if the mechanism grows too large especially if spatial
transport or multi-phase behaviour needs to be included.
\newline
\paragraph{}
We have recently developed a new approach to modeling the kinetics of chemical
networks which we term ‘the sum over histories representation’, or SOHR for short.\cite{ch1_IRPC_15_kramer2014following,ch1_IRPC_16_ch3_6_ch4_8_bai2014sum,ch1_IRPC_17_ch4_9_bai2015sum,ch1_IRPC_18_bai2017simulating} This method replaces the usual concentration based kinetics with a representation
based on chemical pathways and can lead to new insight and increased computational
capability when properly applied. A chemical pathway is a sequence of
elementary reactions where the product of one reaction becomes the reagent for the
next reaction. Of course, the notion of the chemical pathway is quite familiar and is
ubiquitous in all areas of chemistry. In kinetics, it has been often employed to provide
mechanistic insight into complex networks of reactions.\cite{ch1_IRPC_19_he2008graph,ch1_IRPC_20_lehmann2004algorithm,ch1_IRPC_21_feng2010dominant,ch1_IRPC_22_chern1990effective} In chemical synthesis,
for another example, a substrate molecule can be successively functionalized by a
sequence of chemical steps. In atmospheric chemistry, a chlorine atom may be followed
through a catalytic cycle that leads to the breakdown of ozone. We note that in these
cases we are following a chemical moiety as it migrates from species to species due to
chemical reactions. In the SOHR, the idea of the chemical pathway becomes the basis of a quantitative
representation of the chemical kinetics. The pathways are explicitly defined by
following a ‘tagged atom’ as it migrates through species space. We find that if all the
relevant chemical pathways and their associated probabilities are known, then it is possible
to directly compute the value of any kinetic observable. The method draws its
motivation from Feynman’s ‘sum over histories’ approach to quantum mechanics.\cite{ch4_26_feynman2010quantum}
As with Feynman path integrals, chemical pathway theory is often more computationally
intensive than traditional approaches but it can provide new insight or numerical
advantage for select problems.\cite{ch1_IRPC_24_makri1999time,ch1_IRPC_25_cao1994formulation,ch1_IRPC_26_berne1986simulation} The key mathematical relation in SOHR is an explicit expression for the probability of an arbitrary chemical path in terms of an
integral of a time-ordered product. This high-dimensional integral can be evaluated
using an efficient Monte Carlo (MC) integration scheme.
\newline
\paragraph{}
We have found it quite useful to visualise the chemical pathways using the methods
of graph theory. On a chemical graph, the species are represented by nodes (or vertices)
and the chemical reactions that interconvert the species are represented by edges (or
lines).\cite{ch1_IRPC_27_christiansen1953elucidation,ch1_IRPC_28_hansen1988chemical,ch1_IRPC_29_balaban1976chemical,ch1_IRPC_30_temkin1992application,ch1_IRPC_31_lu2005directed} As we follow a tagged atom through the chemical network, the pathway
is represented by a sequence of vertices and edges. The graph then allows any pathway
to be associated with a unique symbol sequence that labels the vertices and edges. Furthermore,
all paths of a given length can be automatically generated by expansions in
the symbol sequence. The chemical graphs we use here are weighted since the reaction
rates can be used to assign a weight to each edge. This identification opens up the possibility
of using graphical search algorithms to locate the most important paths that contribute
to a given process.
\newline
\paragraph{}
One aspect of the present method that distinguishes it from previous attempts to
quantify chemical pathways\cite{ch1_IRPC_20_lehmann2004algorithm,ch3_17_kee2008chemkin} is that SOHR fully embraces the dynamical nature
of the chemical kinetics graphs. The time-dependence of edge weights (i.e. reactions
rates) are accurately included even though this makes the computation of the probabilities
much more challenging. Thus, as a molecule moves through the network its path
instantaneously responds to the time varying rates. Other methods use static snapshots
of the reaction flux or time-averages of the reaction rates to compute quantities such as
the branching ratios in the chemical network. Since the rates of reaction can vary dramatically
as a function of time, the passage through the network may be badly approximated
by the static approximation.
\newline
\paragraph{}
The mathematical underpinning of the SOHR is discussed in Chapter \ref{mathchapter}. The computation
of pathway probabilities and the methods for pathway enumeration are discussed
in detail in this section. In Chapter \ref{chapter:linear_kinetics} we present the results for a chemical network
described by linear kinetics which serve to illustrate some of the general characteristics
of the method. For a chemical system composed only of first-order reactions, the pathway
probabilities can be computed analytically and is found to reduce to a sum of
exponentials in time. It is shown that this linear kinetic problem can be solved with
many fewer paths than might be expected. In Chapter \ref{chapter3}, the H$_2$
combustion problem is analyzed using the SOHR method. It is shown that the ignition
chemistry for this system can be accurately represented using a small number of chemical
pathways. The use of the pathway representation also provides new insight into the
mechanism and kinetic sensitivities of this problem. Chapter \ref{chapter4} demonstrates the applicability of SOHR in predicting chemical kinetics. Finally, Chapter \ref{chapter5} presents a short
conclusion and some ideas for future research.

