\chapter{Conclusion}
\label{chapter5}
The sum over histories representation provides a quantitative approach to investigate
the contributions from distinct chemical pathways in determining the chemistry of large
reaction networks. The basic concept motivating the SOHR is that the value of an
observable target in a kinetic system may be computed from a knowledge of all the
chemical pathways leading to that target. Thus, if we know the pathway probabilities
$P_j$, then the observable can be composed from a linear combination $\sum{c_jP_j}$ where $c_j$ are
trivial coefficients obtained from the rate equations. The implementation of the SOHR
thus involves two main tasks: (1) enumerating the most important chemical pathways
associated with a given observable, and (2) efficiently computing the probabilities of
those pathways. The chemical pathways were defined using a atom-following approach.
We have shown that for sufficiently simple problems an enumeration of paths based on
graph theoretical methods can converge the expansion of observables by increasing
path length. For complicated mechanisms, a more efficient approach utilizes random
walks through the reaction network to identify the most important chemical paths. Once
the pathways have been located, the associated probabilities are computed using the
time-ordered product, eqn. \ref{ch2:eqn13}. This formula was derived assuming that the
reactions along the pathway form a Markov chain of random events. The integral of
the time-ordered product can be evaluated analytically for linear kinetics or for nonlinear
systems in steady state. For more general nonlinear systems, the expression is efficiently
evaluated using a MC-integration scheme, eqn. \ref{ch2:eqn14}. The SOHR method
differs from other techniques for reaction path analysis which represent the pathways at
an instant of time using a flux snapshot.\cite{ch3_17_kee2008chemkin} Instead, our method permits the time
evolving chemistry of the system to be quantitatively incorporated into the probabilities
for the chemical pathways. Hence, the ranking of the chemical pathways followed by
chemical moieties are allowed to change as time progresses.
\newline
\paragraph{}
The SOHR provides a new tool to interpret the quantitative kinetics of large networks
of reactions. Since chemical pathways are, in essence, mechanisms for the creation
of specified species, this method provides an attractive physical picture to
understand the behaviour of large models. In the sum over histories representation, the concentrations of the chemical species are
decomposed into the sum of probabilities for chemical pathways that follow
molecules from reactants to products or intermediates. Unlike static flux
methods for reaction path analysis, the sum over histories approach includes the
explicit time dependence of the pathway probabilities. Using the sum over
histories representation, the sensitivity of an observable with respect to a kinetic
parameter such as a rate coefficient is then analyzed in terms of how that
parameter affects the chemical pathway probabilities. The method is illustrated
for species concentration target functions in H$_2$ combustion where the rate
coefficients are allowed to vary over their associated uncertainty ranges. As we have shown, e.g. the sensitivity of
target functions to rate coefficients in kinetic simulations can have a physically appealing
interpretation in terms of the behaviour of chemical pathways. The reaction can
prove to be sensitive if it is a rate limiting step on a dominant pathway to the target.
Kinetic simplification\cite{ch1_IRPC_67_skodje2001geometrical,ch1_IRPC_68_law2003development,ch1_IRPC_69_maas1992simplifying} of large mechanisms may likewise be guided by a
knowledge of important (or unimportant) paths leading to products.
\newline
\paragraph{}
In addition to interpreting chemical kinetics, we demonstrated SOHR can be developed into an
independent computational technique that is not dependent upon other representations 
of the kinetics. This newly developed method provides an alternative to conventional modeling of mass-action kinetics that involves solving differential equations for the species concentrations.  The method presented in this thesis avoids the need to solve the rate equations by switching to a representation based on chemical pathways.  In the SOHR method, any time-dependent kinetic observable, such as concentration, is written as a linear combination of probabilities for chemical pathways leading to a desired outcome.  In this work, an iterative method is introduced that allows the time-dependent pathway probabilities to be generated from a knowledge of the elementary rate coefficients thus avoiding the pitfalls involved in solving the differential equations of kinetics.  The method was successfully applied to the model Lotka-Volterra system and to a realistic H$_2-$O$_2$ combustion model.