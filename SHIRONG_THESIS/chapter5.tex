\chapter{Conclusion}
\label{chapter5}
The sum over histories representation provides a quantitative approach to investigate
the contributions from distinct chemical pathways in determining the chemistry of large
reaction networks. The basic concept motivating the SOHR is that the value of an
observable target in a kinetic system may be computed from a knowledge of all the
chemical pathways leading to that target. Thus, if we know the pathway probabilities
$P_j$, then the observable can be composed from a linear combination $\sum{c_jP_j}$ where $c_j$ are
trivial coefficients obtained from the rate equations. The implementation of the SOHR
thus involves two main tasks: (1) enumerating the most important chemical pathways
associated with a given observable, and (2) efficiently computing the probabilities of
those pathways. The chemical pathways were defined using a followed-atom approach.
We have shown that for sufficiently simple problems an enumeration of paths based on
graph theoretical methods can converge the expansion of observables by increasing
path length. For complicated mechanisms, a more efficient approach utilizes random
walks through the reaction network to identify the most important chemical paths. Once
the pathways have been located, the associated probabilities are computed using the
time-ordered product, eqn. \ref{ch2:eqn13}. This formula was derived assuming that the
reactions along the pathway form a Markov chain of random events. The integral of
the time-ordered product can be evaluated analytically for linear kinetics or for nonlinear
systems in steady state. For more general nonlinear systems, the expression is efficiently
evaluated using a MC-integration scheme, eqn. \ref{ch2:eqn14}. The SOHR method
differs from other techniques for reaction path analysis which represent the pathways at
an instant of time using a flux snapshot.\cite{ch1_IRPC_32_chemkin201115112} Instead, our method permits the time
evolving chemistry of the system to be quantitatively incorporated into the probabilities
for the chemical pathways. Hence, the ranking of the chemical pathways followed by
chemical moieties are allowed to change as time progresses.
\newline
\paragraph{}
The SOHR provides a new tool to interpret the quantitative kinetics of large networks
of reactions. Since chemical pathways are, in essence, mechanisms for the creation
of specified species, this method provides an attractive physical picture to
understand the behaviour of large models. As we have shown, e.g. the sensitivity of
target functions to rate coefficients in kinetic simulations can have a physically appealing
interpretation in terms of the behaviour of chemical pathways. The reaction can
prove to be sensitive if it is a rate limiting step on a dominant pathway to the target.
Kinetic simplification\cite{ch1_IRPC_67_skodje2001geometrical,ch1_IRPC_68_law2003development,ch1_IRPC_69_maas1992simplifying} of large mechanisms may likewise be guided by a
knowledge of important (or unimportant) paths leading to products. Furthermore, as
illustrated by surface catalysis systems considered above, the chemical pathways themselves
may leave distinct imprints on the observed kinetics.
\newline
\paragraph{}
A particularly exciting prospect is that the SOHR can be developed into an
independent computational technique that is not dependent upon other representations 
of the kinetics. At present, the SOHR method relies on the use of a reference trajectory
$\mathbf{X}(t)$ to compute the pseudo-first-order rate coefficients needed to evaluate the pathway
probabilities. We are exploring the possibility of eliminating this dependency with a
method based on the use of numerical iteration.\cite{ch1_IRPC_18_bai2017simulating} Recall that the reference trajectory
was used in only one way in computing the pathway probability, viz. to determine the
time-dependence of the pseudo-first-order rate coefficients $A_{i,j}(t)$, in eqn. \ref{ch2:eqn11}. If,
instead, $A_{i,j}(t)$ were approximated by constants (i.e. a linear approximation) or by using
the steady state approximation, we would obtain a guess for the pathway probabilities
using eqn. \ref{ch2:eqn13}. That guess could then be used to approximate the concentrations
via eqn. \ref{ch2:eqn16}. Then, the next iteration is generated when the updated concentrations
are used to again compute $A_{i,j}(t)$. Somewhat more formally we can write the iteration
as the abstract forms of eqns. \ref{ch2:eqn13} and \ref{ch2:eqn16}
\begin{equation*}
{\overline{\mathbf{P}}}^{m+1}(t_0,t) = \overline{\mathbf{F}} \left[ {\mathbf{X}}^{m} \right]
\end{equation*}
\begin{equation*}
{\mathbf{X}}^{m+1}(t) = \overline{\overline{\mathbf{G}}} \cdot {\overline{\mathbf{P}}}^{m}(t_0,t)
\end{equation*}
where $\overline{\overline{\mathbf{G}}}$ represents the linear combination of eqn. \ref{ch2:eqn16} and ${\overline{\mathbf{P}}}$ represents the
time-ordered integral in eqn. \ref{ch2:eqn13} and $m$ is the iteration number.